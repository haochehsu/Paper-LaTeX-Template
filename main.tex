\documentclass[letterpaper]{article}
\usepackage[colorlinks=true,linkcolor=Blue,urlcolor=Blue,citecolor=Blue,anchorcolor=Blue]{hyperref}
\usepackage{Style/arxiv, Style/package}

%%%%%%%%%% ABBREVIATIONS %%%%%%%%%%
\newcommand{\R}{\mathbb{R}}
\newcommand{\h}{\mathcal{H}}
\newcommand{\E}{\mathbb{E}}
\newcommand{\Pn}{\mathbb{P}_n}
\newcommand{\bs}{\boldsymbol}
\newcommand{\sgn}{\mathrm{sign}}
\newcommand{\diag}{\mathrm{diag}}
\newcommand{\F}{\mathcal{F}}
\newcommand{\trace}{\mathrm{trace}}
\newcommand{\cov}{\mathrm{cov}}
\newcommand{\var}{\mathrm{var}}
\newcommand{\A}{\mathcal{A}}
\newcommand{\B}{\mathcal{B}}
\newcommand{\x}{{\bs X}}
\newcommand{\g}{\cellcolor{gray!25}}
\newcommand{\gline}{\rowcolor{gray!25}}
\newtheorem{theorem}{Theorem}
\newtheorem*{theorem*}{Theorem}
\newtheorem*{definition}{Definition}
\newtheorem*{assumption}{Assumption}
\newtheorem{lemma}{Lemma}
\newtheorem{example}{Example}
\newtheorem{corollary}{Corollary}
\newtheorem{remark}{Remark}
\newtheorem{proposition}{Proposition}
\renewcommand{\headeright}{~}
\renewcommand{\undertitle}{~}
\renewcommand{\shorttitle}{~}
\def\lquote{\textquotedblleft}
\def\rquote{\textquotedblright \ }

%%%%%%%%%% HEADER & FOOTER %%%%%%%%%%
\def\target_page{1}
\fancypagestyle{page_style}{\fancyhf{}
\renewcommand{\headrulewidth}{0pt} % 0.4pt
% \fancyhead[R]{Top Right Header}
% \fancyhead[L]{Top Left Header}
% \chead{Top Center Header}
\renewcommand{\footrulewidth}{0pt} % 0.4pt
% \fancyfoot[R]{Bottom Right Footer}
% \fancyfoot[L]{Bottom Left Footer}
% \fancyfoot[C]{\thepage}
\fancyfoot[C]{\hyperlink{page.\target_page}{{\color{black}\thepage}}}}
\pagestyle{page_style}

%%%%%%%%%% CUSTOMIZE %%%%%%%%%%
\sisetup{parse-numbers = false}
\renewcommand*\thefootnote{\textcolor{red}{\arabic{footnote}}}
\definecolor{mycolor}{HTML}{3D8A2B}
\definecolor{darkerGreen}{HTML}{278041}
\definecolor{darkerPurple}{HTML}{744CB4}
\definecolor{clay}{HTML}{CA7C20}

%%%%%%%%%% LISTING %%%%%%%%%%
\lstdefinestyle{python_code}{
language=Python, % python, matlab, r
basicstyle=\linespread{1}\footnotesize\ttfamily,
numbers=left,
numberstyle=\tiny,
frame=tb,
columns=fullflexible,
showstringspaces=false,
breaklines=true}

%%%%%%%%%% FONT SIZE (DEFAULT: 10pt) %%%%%%%%%%
\makeatletter
\input{size11.clo} % 11pt
% \input{size12.clo} % 12pt
\makeatother

%%%%%%%%%% Title %%%%%%%%%%
\title{The Paper Title\thanks{We thank someone for excellent research assistance. We thank someone for their comments and suggestions.}}

%%%%%%%%%% AUTHOR %%%%%%%%%%
\author{\href{}{Your Name}\thanks{Your address; Email: \href{mailto:yourEmail@email.com}{yourEmail@email.com}.}\\
Department of Economics}

% UNCOMMENT BELOW IF THERE ARE MULTIPLE AUTHORS
% \author{
% Your Name\thanks{Your address; Email: \href{mailto:yourEmail@email.com}{yourEmail@email.com}.}\\Department of Economics
% \And
% Your Coauthor\\Department of Economics
% }

%%%%%%%%%% DATE %%%%%%%%%%
\date{\the\year{}}

%%%%%%%%%% BEGIN DOCUMENT %%%%%%%%%%
\begin{document}
\maketitle

% \singlespacing
% \onehalfspacing
\doublespacing

%%%%%%%%%% ABSTRACT %%%%%%%%%%
\begin{abstract}
  \onehalfspacing
  \lipsum[3] Download this template at the \href{https://github.com/howardhsumail/Paper-LaTeX-Template.git}{Github repository}.
\end{abstract}
\bigskip\bigskip

%%%%%%%%%% KEYWORDS & CODES %%%%%%%%%%
\providecommand{\key}[1]{\textbf{Keywords:} #1}
\key{keyword1, keyword2}

\providecommand{\jel}[2]{\textbf{JEL Codes:} #1}
\jel{J02, R10}

%%%%%%%%%% CONTENTS %%%%%%%%%%
\newpage
\section{Introduction}
\lipsum[4-5] Many previous research has has studied this problem \citep{Lee2018, DS2018}. Download this template at the \href{https://github.com/howardhsumail/Paper-LaTeX-Template.git}{Github repository}.

\section{Model}
Equation \ref{myEquation} or \cref{myEquation} or Equation \eqref{myEquation}. \lipsum[105-106] More are discussed in \Cref{discussion}.

\begin{theorem}[Envelope Theorem]\hlabel{T1}
  \onehalfspacing
  Only the direct effects of a change in an exogenous variable need be considered, even though the exogenous variable may enter the maximum value function indirectly as part of the solution to the endogenous choice variables. The proof is in Appendix \ref{proof_T1}.
\end{theorem}

The {\color{mycolor}competition} can be illustrated with the following graph with the implementation is presented in Listing \ref{mypythoncode} or \cref{mypythoncode}:

\begin{figure}[H]
  \centering
  \caption{This is a graph}
  \hlabel{picture1}
  \includegraphics[scale=0.5]{Graph/pic.pdf}
  \hspace*{-0.6cm}
  \begin{minipage}{0.9\textwidth}
    \onehalfspacing
    \vspace*{0.12cm}
    \begin{tablenotes}
      \footnotesize
      \item\textit{Note:} some notes. The graph should be self-contained. \lipsum[65]
    \end{tablenotes}
  \end{minipage}
\end{figure}

\section{Comparative Statics}

This is also demonstrated in Figure \ref{picture1} or \cref{picture1} and the results are presented in Appendix \ref{summary_b}. Download this template at the \href{https://github.com/howardhsumail/Paper-LaTeX-Template.git}{Github repository}. \lipsum[104] 

\begin{lstlisting}[style=python_code, caption={Long short-term memory}, label=mypythoncode]
class network_LSTM(nn.Module):
    def __init__(self, input_size=1, hidden_size=256, output_size=1):
        super().__init__()
        self.hidden_size = hidden_size
        self.lstm = nn.LSTM(input_size, hidden_size)

        # fully-connected
        self.linear = nn.Linear(hidden_size, output_size)

        self.hidden = (
            torch.zeros(1, 1, self.hidden_size),
            torch.zeros(1, 1, self.hidden_size)
        )

    def forward(self,vec):
        lstm_output, self.hidden = self.lstm(vec.view(len(vec),1,-1), self.hidden)
        prediction = self.linear(lstm_output.view(len(vec),-1))
        return prediction[-1]
\end{lstlisting}

\section{Empirical Results}
We follow the approach from \cite{HL2019}. By using this approach, comparable results can be obtained \citep{CES2013}. To calculate the ELBO\footnote{More information about the evidence lower bound (ELBO) can be found on the \href{https://en.wikipedia.org/wiki/Evidence_lower_bound}{Wikipedia}. }, we start from using the property of the KL-divergence. The data can be summarized by the tables with decimal alignment in Table \cref{summary_a}.

\lipsum[107-109]

\renewcommand*\arraystretch{1.5}
\renewcommand{\tabcolsep}{2.5pt}
\begin{table}[H]
  \centering
  \renewcommand{\thetable}{B.\arabic{table}a}
  \caption{First Table}
  \label{summary_a}
  \fontsize{10}{11}\selectfont
  \hspace*{0cm}\begin{tabular}{lrrrrrrrrr}
  \toprule
  Category                   & Total & Shares (\%) & Female & Male  & Asian & Black/AA & His./Latino & White/Cau. & Zeros (\%) \\ \hline
  child care                 & 19.39 & 0.08   & 12.32  & 20.12 & 23.14 & 63.78    & 20.24       & 19.00      & 0.07  \\
  eating                     & 30.35 & 6.12   & 35.97  & 6.23 & 24.61 & 21.58    & 38.18       & 2.02      & 0.00  \\
  education                  & 9.91  & 0.04   & 9.94   & 90.54  & 9.69  & 7.99     & 10.64        & 10.14      & 0.90  \\
  entertainment (not TV)     & 26.05 & 0.10   & 29.19  & 26.60 & 33.36 & 26.13    & 4.43       & 25.15      & 0.45  \\ \bottomrule
\end{tabular}
\hspace*{-0.7cm}
\begin{minipage}{1\textwidth}
  \onehalfspacing
  \vspace*{0.05cm}
  \begin{tablenotes}
    \footnotesize
    \item\textit{Note:} This is the first table.
  \end{tablenotes}
\end{minipage}
\end{table}

\renewcommand*\arraystretch{1.5}
\renewcommand{\tabcolsep}{4.1pt}
\begin{table}[H]
  \centering
  \addtocounter{table}{-1}
  \renewcommand{\thetable}{B.\arabic{table}b}
  \caption{Second Table}
  \label{summary_b}
  \fontsize{10}{11}\selectfont
  \hspace*{0cm}\begin{tabular}{lrrrrrrrrr}
  \toprule
  Category                   & Total & Shares (\%) & Female & Male  & Asian & Black/AA & His./Latino & White/Cau. & Zeros (\%) \\ \hline
  child care                 & 19.39 & 0.08   & 39.32  & 40.12 & 23.14 & 18.78    & 20.24       & 19.00      & 0.07  \\
  personal care              & 13.92 & 0.06   & 24.00  & 23.14 & 16.12 & 1.76    & 15.15       & 13.66      & 0.00  \\
  sports/exercise            & 20.44 & 0.08   & 20.38  & 31.00 & 24.99 & 25.48    & 20.71       & 20.07      & 0.53  \\
  TV                         & 28.61 & 0.12   & 48.47  & 9.93 & 2.35 & 63.70    & 29.22       & 80.20      & 0.46  \\ \bottomrule
\end{tabular}
\hspace*{-0.6cm}
\begin{minipage}{1\textwidth}
  \onehalfspacing
  \vspace*{0.05cm}
  \begin{tablenotes}
    \footnotesize
    \item\textit{Note:} This is the second table.
  \end{tablenotes}
\end{minipage}
\end{table}

\section{Algorithm}

\lipsum[50] In the following, we present the algorithm:

\begin{algorithm}
  \SetKwInOut{Input}{Input}
  \SetKwInOut{Output}{Output}

  \underline{function Euclid} $(a,b)$\;
  \Input{Two nonnegative integers $a$ and $b$}
  \Output{$\gcd(a,b)$}
  \eIf{$b=0$}
  {
  return $a$\;
  }
  {
  return Euclid$(b,a\mod b)$\;
  }
  \caption{Euclid's algorithm for finding the greatest common divisor of two nonnegative integers}
\end{algorithm}

\lipsum[51]

\renewcommand*\arraystretch{1.2}
\renewcommand{\tabcolsep}{25pt}
\begin{table}[H]
  \centering
  \caption{Summary Statistics}
  \hlabel{ss}
  \fontsize{10}{11}\selectfont
  \begin{tabular}{
    l
    *{5}{S[table-format=2.1]}
    }

    \toprule
    & \multicolumn{3}{c}{\bfseries Cohort} \\
    \cmidrule(l){2-4}
    & {2006} & {2007} & {2008} \\
    \midrule
    \bfseries Students registered & {1535} & {1584} & {1767}\\
    \addlinespace
    \bfseries Gender (\%) \\
    Male                         & 61.1 & 64.5 & 57.7\\
    Female                       & 38.9 & 35.5 & 42.3\\
    \addlinespace
    \bfseries Race (\%) \\
    White                        & 43.3 & 43.4 & 40.6\\
    Black                        & 29.8 & 33.4 & 34.8\\
    \bottomrule
  \end{tabular}
  \begin{minipage}{0.79\textwidth}
    \onehalfspacing
    \vspace*{0.05cm}
    \begin{tablenotes}
      \footnotesize
      \item\textit{Note:} Source: UCT Institutional Planning Department.
    \end{tablenotes}
  \end{minipage}
\end{table}

\begin{lemma}[This is a lemma]\label{lma1}
  \onehalfspacing
  \begin{enumerate}[(a)]
    \setlength{\itemsep}{-0.5pt}
    \item[] \hfill
    \item For any feasible disclosure policy  $G\in \mathcal{G}$, $W_G(p)$ is a convex function. Moreover, for all $p\in[0,1]$, $W_{G_{\underline{\pi}}}(p)\leq W_G(p)\leq W_{G_{\overline{\pi}}(p)}$.
    \item The converse of the above statement is also true. That is, if $W:[0,1]\to \mathbbm{R}$ is a convex function that satisfies $W_{G_{\underline{\pi}}}(p)\leq W(p)\leq W_{G_{\overline{\pi}}(p)}$, then there exists a feasible $G\in\mathcal{G}$ such that $W_G(p) = W(p)$ for all $p\in[0,1]$.
  \end{enumerate}
\end{lemma}

\section{Conclusion}

\lipsum[14-15] We graph with \texttt{tikz} in \LaTeX:

\begin{multicols}{2}
  \raggedcolumns

  \begin{tikzpicture}[scale=.65]
    \draw [<-] (0,7.8) node [left] {$x_2,y_2$} -- (0,0);
    \draw [->] (0,0) -- (8.9,0) node [below] {$x_1,y_1$};
    \node [left] at (0,6.45) {$P$};
    \node [below] at (5.05,0) {$P'$};
    \node [left] at (4.3,3.2) {$x^n$};
    \node [below] at (2.4,5.4) {$y^f$};
    \node [right] at (5.75,3.4) {$x^f$};
    \draw (5.8,.5) -- (2.65,6.7);
    \draw (.5,7.2) -- (7.7,1.7);
    \draw (4.45,5.3) to [out=-90, in=140] (5.75,3.2) to [out=-40, in=160] (7.7,2.2);
    \draw (4.15,4.3) to [out=-90, in=120] (4.55,3) to [out=-60, in=160] (7.4,1.1);
    \draw (0,6.45) to [out=0, in=115] (4.4,3.2) to [out=-65, in=90] (5.05,0);
  \end{tikzpicture}

  \columnbreak

  \begin{tikzpicture}[scale=0.98]\hlabel{triangle}
    \draw [thick] (0,0) node [left] {$\Pi_3$} -- (3,1.8) node [below] {0} -- (3,5.4) node [right] {$\Pi_2$};
    \draw [thick] (3,1.8) -- (6,0) node [right] {$\Pi_1$};
    \draw [thick] (.4,.26) -- (3,4.8);
    \draw [thick] (.4,.26) -- (5.55,.26) -- (3,4.8);
    \node [thick,below] at (1,.26) {$(0,0,1)$};
    \node [thick,right] at (3,5) {$(0,1,0)$};
    \node [thick,right] at (5.65,.4) {$(1,0,0)$};
  \end{tikzpicture}

\end{multicols}

\clearpage

%%%%%%%%%% REFERENCES %%%%%%%%%%
\bibliographystyle{Style/aea} % Style/ecta
\phantomsection
\addcontentsline{toc}{section}{\refname}
\bibliography{reference}

%%%%%%%%%% APPENDIX %%%%%%%%%%
\newpage
\appendixpage
\appendix

\section{Additional Discussion}\hlabel{discussion}

\lipsum[103]

\section{Proof of \Cref{T1}}\hlabel{proof_T1}

\hlabel{ch1_proof}We will proof the following equation:
\begin{proof}

  Given $y$, $x$, $\Delta$, $\nu$, $\eta$, $\mathcal{L}$=
  $\begin{pmatrix}
  1 & 2 & 3 & 4 & 5 \\
  3 & 4 & 5 & 6 & 7
  \end{pmatrix}$,
  and $\prod=\begin{vmatrix}
  A &B  &C \\
  D&  E& F
  \end{vmatrix}$, if

  \begin{center}
    \begin{tabular}{ll}
      $\begin{cases}
      \text{trade}, & p(\text{trade})=\dfrac{y}{v}\\
      \text{no trade}, & p(\text{no trade})=1-\dfrac{y}{v}
      \end{cases}$
    \end{tabular}
  \end{center}
  
  \vspace*{-1cm}

  \begin{align}\label{myEquation}
    \nonumber y&=\underset{\pi}{\E}\Big(\beta x + \epsilon\Big)\neq\sum\limits_{i}\beta_i(\underbrace{\alpha+\xi}_{\text{variables}}) + \epsilon\\
    &\Longrightarrow \int_{0}^{10}r \left( \dfrac{r}{50} \right)dr\xlongequal{\text{text here}}\dfrac{r^{3}}{150}\biggr\rvert^{10}_{0}, \forall x\in (a,b)
  \end{align}
  So from $\widehat{ABCD}$, $\widetilde{ABCD}$, $\widehat{ABCD}$, $\overrightarrow{ABCD}$, and $\overline{ABCD}$, we get the desire $\underline{\text{result}}$.
\end{proof}

\begin{framed}
  Consider $g(x)=f(x)-x$, since $f(x)$ and $x$ are continuous, then $g:[a,b]\to\mathbb{R}$ is continuous. Then
  $$g(a)=f(a)-a>0, \ g(b)=f(b)-b<0$$
  By IVT: $\exists c\in(a,b)$ s.t. $g(c)=0\implies \exists c\in(a,b)$ s.t. $f(c)-c=0\implies f(c)=c.$
\end{framed}

\section{More \texttt{Tikz}}\hlabel{tikzpics}

\begin{figure}[H]
  \centering
  \caption{Caption above figure}
  \begin{tikzpicture}[xscale=1, yscale=1.5]
    \draw[line width=0.35mm, ->] (-0.1,0) -- (8.2,0) node[right] {$p$};
    
    \draw[line width=0.35mm, ->] (0,-0.1) -- (0,4.2) node[above] {$W_G(p)$};
    
    \draw (4,0) node[below ]{$\mu$};
    \draw (0,4) node[left]{$\mu$};
    
    \draw[line width=0.3mm, blue, name path =A] (0,0) plot[domain=0:8] (\x,{1/16*(8-\x)^2}) node[anchor=south] at (8,4) {$W_{\overline{\pi}}(p)$};
    \draw[line width=0.3mm,red, name path =B] (0,4) -- (4,0) -- (8,0) node[anchor=north] at (8,4) {$W_{\underline{\pi}}(p)$};
    
    \draw[line width=0.3mm, orange, name path =C](0,0)
    plot[domain=0:6.82746] (\x,{4-1.08586 *\x + 0.0732323*\x^2}) node[anchor=north] at (8,2.5){$W_G(p)$};
    
    \draw[line width=0.3mm, orange, name path =D](6.82746,0)
    -- (8,0);
    
    \draw (4.4,0) node[below]{$p_0$};

    \node[outer sep=0pt, circle, fill=darkerPurple, inner sep=1pt, minimum size=1.5mm] (P) at
    (4.4,0.64) {};

    \draw[line width=0.3mm, darkerGreen] (8,0) -- (5.37538,0) -- (P) -- (2.75076,1.7221575361) -- (1.37539,2.6241)--(0,4) node[anchor=south] at (8,2.5){$W_{G^\prime}(p)$};

    \node[outer sep=0pt, circle, fill=clay, inner sep=1pt, minimum size=1.5mm] (Q1) at
    (2.75076,0) {};

    \draw (Q1) node[below]{$\gamma_0$};

    \node[outer sep=0pt, circle, fill=darkerPurple, inner sep=1pt, minimum size=1.5mm] (Q2) at
    (2.75076,1.7221575361) {};

    \path[draw=red, dashed] (Q1) -- (Q2);
    \path[draw=red, dashed] (4.4,0) -- (P);
  \end{tikzpicture}
  \label{fig:2}
\end{figure}

\begin{figure}[H]
  \centering
  \begin{tikzpicture}[xscale=10,yscale=5,
    userDefineLine/.style={line width=0.4mm, red}
    ]
    \draw[line width=0.3mm, ->] (0,0) -- (1.04,0);
    \node (B) at (1.1,0) {\text{prior}};
    
    \draw[line width=0.3mm, ->] (0,0) -- (0,0.8);
    \node (B) at (0.32,0.32) {$V_1^*(p^0)$};
    \draw (0,0) node[left]{\footnotesize $0$};
  
    \draw[userDefineLine] (0,0) -- (0.3,0);
    \draw[userDefineLine] (0.3, 0) -- (1,0.7);
    
    \draw[line width=0.3mm, red, dotted] (0.5,0) -- (0.5, 0.35);
    
    \draw[line width=0.6mm, violet, opacity=0.7, dashed] (0,0) -- (1,0.7);

    % x axis
    \draw (0.3,0) coordinate(g1) node[below ]{$p$};
    \draw (g1) ++(0,-0.01) -- ++(0,0.02);
    
    \draw (1,0) coordinate(g2) node[below]{$1$};
    \draw (g2) ++(0,-0.01) -- ++(0,0.02);
    
    \draw[dotted] (1,0) -- (1,0.7);
    
    % y axis
    \draw (0,0.7) coordinate(u1) node[left]{\footnotesize $1-p$};
    \draw (u1) ++(-0.01,0) -- ++(0.02,0);
    
    \draw[dotted] (u1) -- (1,0.7);
    
    %Legend
    \matrix [draw, fill=white, below right] at (1.05,0.5) {
    \draw [userDefineLine] ++(-0.3,0) -- ++(0.6,0) node[black,right] {$V_1(p^0)$}; \\
    \draw [line width=0.6mm, violet, opacity=0.7, dashed] ++(-0.3,0) -- ++(0.6,0) node[black,right,opacity=1] {$V_1^*(p^0)$}; \\
    };
    
    \node[outer sep=0pt, circle, fill=blue, inner sep=1pt, minimum size=1.5mm] (P) at
    (0.5,0.35) {};
    
    \node (C) at (0.5,-0.06) {\color{blue}$p^0=\frac{1}{2}$};
    
  \end{tikzpicture}
  \caption{Caption below figure} \label{fig:M2}
\end{figure}

\begin{figure}[H]
  \centering
  \begin{tikzpicture}[scale=10.0, xscale=2]
    \draw[line width=0.3mm, ->] (1/2, 0) -- (1, 0) node[right] {$\lambda_i$};
    \draw[line width=0.3mm, ->] (1/2, 0) -- (1/2, 1/3) node[above] {$R_j$};
    
    % blue lines
    \draw[line width=0.3mm, scale=1, domain=1/2:11/16, smooth, variable=\x, blue] plot ({\x}, {1/5});
    \draw[line width=0.3mm, scale=1, domain=11/16:1, smooth, variable=\x, blue] plot ({\x}, {1/4-1/8-1/2*(1/8-((2*\x-1)/2)*(2/3)});
    
    % red lines
    \draw[line width=0.3mm, scale=1, domain=1/2:5/8, smooth, variable=\x, red] plot ({\x}, {1/4});
    \draw[line width=0.3mm, scale=1, domain=5/8:1, smooth, variable=\x, red] plot ({\x}, {1/4-1/8-1/2*(1/8-((2*\x-1)/2))});
    
    \draw[line width=0.3mm, dashed,black,thick] (5/8,1/4)--(1,1/4);
    \draw[line width=0.3mm, dashed,black,thick] (11/16,1/5)--(1,1/5);
    
    \node[circle,inner sep=1pt,fill=black,label=left:{$\frac{1}{4}$}] at (1/2,1/4) {};
    \node[circle,inner sep=1pt,fill=red,minimum size=2mm,label=below:{$\frac{5}{8}$}] at (5/8,0) {};
    \node[circle,inner sep=1pt,fill=blue,minimum size=2mm, label=below:{$
    \frac{11}{16}$}] at (11/16,0) {};
    \node[circle,inner sep=1pt,fill=black,label=below:{$\frac{1}{2}$}] at (1/2,0) {};
    
    \matrix [draw, above left] at (1.11,0.08) {
    &\node[red,font=\fontsize{10}{10}\selectfont] {$\mu=1$}; \\
    &\node[blue,font=\fontsize{10}{10}\selectfont] {$\mu=2/3$}; \\
    };
    
    \node (D) at (0.8,-0.04) {$p=\frac{1}{4}$};
  \end{tikzpicture}
  \caption{Caption below figure} \label{fig:M3}
\end{figure}

\end{document}